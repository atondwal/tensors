\documentclass[11pt,notitlepage]{article}
\usepackage{amsfonts, amsmath, amssymb, amsthm,fullpage,mdwlist,graphicx,cancel}
\setlength{\textheight}{9.25in}
\setlength{\textwidth}{6.5in}
\setlength{\topmargin}{0.0in}
\setlength{\headheight}{0.0in}
\setlength{\headsep}{0.0in}
\setlength{\leftmargin}{0.0in}
\setlength{\oddsidemargin}{0.0in}
\setlength{\parindent}{0pc}
\everymath{\displaystyle}
\newtheorem{thm}{Theorem}[section]
\newtheorem{exc}{Exercise}[section]
\title{Tensors and Co-/Contra- Variance}
\author{Anish Tondwalkar}
\date{\today}
\begin{document}
\maketitle
In physics, all of our measurements are going to correspond to scalars (specifically, real numbers). 
Furthermore, because our physical measurements can't depend on our mathematical conventions and choice of coordinate system,
these scalars much be the same (invariant) under rotations of the coordinate system.
\section{Contravariance}
Vectors, however, aren't tied to the events we're measuring by their coordinates in our coordinate system, but by their geometric meaning.
We can see, by trivial example that a vector transforms `away' from our rotation. 
Thus we will call things that transform this way `contravariant'
\footnote{Indices for contravariant vectors go at the top. This is different from the lower index notation you've probably been using all your life, because in Euclidean spaces, the covariant and contravariant form of the vector are the same, so we can just write everything covariantly, because it's less ambiguous notation}. 
\begin{exc}
Find a coordinate system in which
$\hat\i + \hat\j + \hat k$
 can be written as $\sqrt3 \hat\j$
\end{exc}
In order to go from one coordinate system to another, you should know that we use the cosines of the angle between them:
$$ x^i' = \sum_j x^j \cos(x^j,x^i') $$
Written with differentials, we have
$$ dx_i' = \sum_j \frac{\partial x_i'}{\partial x_j} dx_j $$
Giving us our transformation law for contravariant tensors:
$$ A^i' = \sum_j \frac{\partial x_i'}{\partial x_j} A^j $$
\section{Covariance}
But that's not all! We know things that transform the other way: gradients:
$$ \del \psi = \sum_j \frac{\partial \psi}{\partial x_j} \hat x^j $$
 They're our prototype for covariant forms. 
From the chain rule, we have: 
$$\frac{\partial \psi}{\partial x^i'} = \sum_j \frac{\partial \psi}{\partial x^j} \frac{\partial x^j}{\partial x^i'} $$
So out covariant transformation law is:
$$ A_i = \sum_j \frac{\partial x^j}{\partial x^i'} A_j$$
\section{n-forms and tensors}

\end{document}
